\documentclass[12pt]{article}

% Fonts
\usepackage[utf8]{inputenc}
\usepackage[T1]{fontenc}
\usepackage{tgpagella}
\usepackage{xcolor}

% Layout
\usepackage[a4paper, margin = 0.75in]{geometry}
\usepackage{setspace}
\onehalfspacing

% Figs and tables
\usepackage{graphicx}
\usepackage[export]{adjustbox}
\usepackage{float}

% Math
\usepackage{amsmath}
\usepackage{amssymb}
\numberwithin{equation}{section}
\usepackage{breqn}

% Others
\usepackage{hyperref}
\usepackage{eurosym}

\title{Income Statement and Balance Sheet Forecast}
\author{Marco Yuyuan Zhang  \thanks{PhD Candidate in Economics, HKU Business School, The University of Hong Kong. Email: marcoyzhang@gmail.com. This report is intended for the 2025 Machine Learning Center of Excellence Summer Associate – Time Series \& Reinforcement Learning Internship at J.P. Morgan Chase.}}
\date{\today}


\begin{document}

\maketitle
	
\begin{abstract}
	Financial statements are collections of interdependent fields that collectively illustrate a firm's financial position. A robust model for predicting financial statements should adhere to the identities established by accounting principles, such as the equation: assets = liabilities + equity. This report is organized as follows: Section 1 transforms a standard spreadsheet model of the balance sheet and income statement into a time series representation and implements this model using TensorFlow. Section 2 examines the model's performance and explores machine learning techniques that can enhance prediction accuracy.
\end{abstract}

\pagebreak



%[what does forecast balance sheet mean? which variables? forecasted variables should satisfy some constraints (accounting identities)]
A balance sheet consists of three important indicators that aggregate key information representing the financial positions of a firm: assets, liabilities, and equity. Balance sheet forecasting involves predicting the future values of all three variables as well as their important components such as current assets and retained earnings. The balance sheet is designed in a way such that the fundamental accounting identity -- namely, assets = liabilities + equity -- always holds. A reasonable forecast model should respect this and other accounting identities when making predictions. In other words, the predicted output should satisfies some constraints imposed by basic accounting principles.

%[possible to model the evolution of fields of a balance sheet as a time series. how to handle accounting]
Mathematically, a balance sheet can be represented by a time series model:
\begin{equation}
	\label{eq:model}
	y(t+1) = f(x(t), y(t)) + n(t)
\end{equation}
where $y(t)$ is a random vector that collects all fields of a balance sheet. For example, $y(t) = (\text{assets}_{t},\text{liabilities}_{t},\text{equity}_{t})^{'}$ is the most simplified version of a balance sheet\footnote{By default, all vectors are column vectors}. $x(t)$ represents a vector of exogenous variables such as net income and/or cash flow from external financing. Exogenous variables are those that affect the balance sheet but not included in it. $n(t)$ is the noise term, reflecting the uncertainty in forecasting balance sheet fields. The forecasting function, denoted by $f(\cdot,\cdot)$, is the key element we aim to identify. It is crucial that the choice of $f(\cdot,\cdot)$ has to reflect all accounting identities in order to produce a reasonable predictive model.
 
Model \ref{eq:model} represents a one-step-ahead, multi-output problem which is straightforward to implement using Google's TensorFlow library in Python. 
[how to train model]

%[data]
Balance sheet, income statement, and cash flow data are sourced from Yahoo Finance using the yfinance data API. I obtain the financial statements for all companies included in the Nasdaq 100 index -- stocks of 100 largest non-financial companies listed on the Nasdaq Stock Exchange and one of the most widely tracked stock index around the world. Nasdaq 100 has a large concentration in blue chip stocks in technology and biotechnology sector. In other words, a big fraction of companies in Nasdaq 100 are industry-leading tech giants that have more stable financial health compared with small and medium cap companies.

[test forecast ability for balance sheet]

[check forecast respects accounting identities]

[forecast earnings using balance sheet model]

[ML techniques improve model. compare. random walk. linear. LSTM.]



\textbf{Literature.} Velez-Pareja (2010, 2011) provides a detailed description of the accounting identities built in standard financial statements. This balance sheet  model formalizes the spreadsheet model in Velez-Pareja (2010) and Velez-Pareja (2011) as a time series representation and applied machine learning techniques to improve its predictability.



\section{Model}

A simplified balance sheet model comprises four key equations: three that govern the evolution of total assets, liabilities, and equity, and one that represents the fundamental accounting identity: assets = liabilities + equity.

\begin{dmath}
	\label{eq:bs_1}
	\text{Total assets}_{t} = \text{Cash}_{t} + \text{Account receivable}_{t} + \text{Inventory}_{t} + \text{Advance payments paid}_{t} + \text{Short-term investments}_{t} + \text{Net fixed assets}_{t}
\end{dmath}

\begin{dmath}
	\label{eq:bs_2}
	\text{Total liabilities}_{t} = \text{Accounts payable}_{t} + \text{Advance payments received}_{t} + \text{Short-term debt}_{t} + \text{Long-term debt}_{t}
\end{dmath}

\begin{dmath}
	\label{eq:bs_3}
	\text{Total equity}_{t} = \text{Equity investment}_{t} + \text{Retained earnings (accumulated)}_{t} + \text{Net income (current year)}_{t} + \text{Repurchase of equity}_{t}
\end{dmath}

\begin{dmath}
	\text{Total assets}_{t} = \text{Total liabilities}_{t} + \text{Total equity}_{t}
\end{dmath}

This four-equation balance sheet model can be easily cast into Model \ref{eq:model} where  $y(t) = (\text{Total assets}_{t},\text{Total liabilities}_{t},\text{Total equity}_{t})^{'}$ and $x(t)$ consists of all variables on the right hand side from equation \ref{eq:bs_1} to \ref{eq:bs_3}. Lagged values of both $y(t)$ and $x(t)$ are useful for predicting $y(t)$. Appendix B specifies a comprehensive model, as outlined by Velez-Pereja (2010).
+

\subsection*{Forecasting Earnings}
Earnings, or net income, is defined as profits after all expenses, taxes, and costs have been deducted from total revenue \footnote{Henceforth, earnings and net income are used interchangeably.}:
\begin{dmath}
	\text{Net Income}_{t} = \text{Sales revenues}_{t} - \text{Cost of goods sold}_{t} - \text{A\&S expenses}_{t}  - \text{Depreciation}_{t} + \text{Return from short-term investments}_{t} - \text{Interest payments}_{t} - \text{Income Taxes}_{t}
\end{dmath}
\section{Testing and Validation}

[present tests here.]

\section*{Reference}

Velez-Pareja, I. (2010). Constructing Consistent Financial Planning Models for Valuation. IIMS Journal of Management of Science, 1.

Vélez-Pareja, I. (2011). Forecasting Financial Statements with No Plugs and No Circularity. The IUP Journal of Accounting Research \& Audit Practices, 10(1).

\newpage
\appendix
\setcounter{page}{1} % Restart page numbering
% \renewcommand{\thepage}{A\arabic{page}} % Optional: Prefix with 'A' for Appendix
\section*{Appendix} % Add a section title "Appendix" at the beginning
\addcontentsline{toc}{section}{Appendix} % If you want it in the table of contents
% Define prefix for tables and figures in the appendix
\renewcommand{\thetable}{A\arabic{table}} % For tables
\renewcommand{\thefigure}{A\arabic{figure}} % For figures

\section*{Question 2: Income Statement and Balance Sheet Forecast}
\label{sec:question}

\subsection*{Part 1}

1) We would like to forecast the balance sheet of a company. Unfortunately, the different fields of a balance sheet are not independent. Hence we have to construct a model that respects these identities. For a short introduction to the problem, please consider the papers Velez-Pareja(09) and Velez-Pareja(10). For a much more detailed exposition of the problem, please consult Shahnazarian(04) and the textbook “Financial Forecasting, Analysis and Modelling” by Samonas, as well as other standard accounting textbooks.

2) Construct a very simple model of the balance sheet based on the tools of Velez-Pareja(09) and Velez-Pareja(10). Please write down the mathematical equations governing the evolution of the fields of balance sheet. Is it possible to model this problem as a time series? How do we handle the accounting identities?

3) Implement the model in TensorFlow and Python.

4) You can get income statement and balance sheet data from Yahoo Finance. This blog post may help you: \url{https://rfachrizal.medium.com/how-to-obtain-financial-statements-from-stocks-using-yfinance-87c432b803b8}.

5) Choose some companies to apply your model to. How are you going to train your model? How can you test if your model is good at forecasting the balance sheet of the company? How can you ensure that your forecast at least respects the accounting identities and at least satisfies the asset = liability + equity identity as other relationships stated in the papers quoted here?

6) Can you use your model to forecast earnings?

7) What are the ML techniques we can use to make your model better?

8) Hint: simulation is highly related to prediction. Suppose that you can simulate \(y(t+1)\) given \(y(t)\). The prediction problem is very simple to implement numerically. A general form of the model can be written as \(y(t+1) = f(x(t), y(t)) + n(t)\), where \(n(t)\) is some noise term to be specified, and \(x(t)\) are additional sets of variables that are relevant for the simulation. What should \(x(t)\) be? 

\subsection*{Part 2}

a) Choose your favourite LLM to apply the problem of financial statement analysis.

b) Let’s try the task of balance sheet forecast using the same set of data as collected in part 1, does the LLM you picked perform better or worse than your model?

c) Is it possible to combine your model in part 1 and LLM to create an ensemble model that performs better than the individual model in balance sheet forecast?

d) Given the results of your analysis, pick a company you have analysed, what would you recommend to the CFO or CEO of this particular company given your results?

\section*{Full Model}
\label{Full Model}
Table 2 Nominal increase

\begin{equation}
	\text{Selling}_{t} = (1 + \text{Inflation rate}_{t})(1+\text{Real increase in selling price}_{t})-1
\end{equation}


\begin{equation}
	\text{Purchasing}_{t} = (1 + \text{Inflation rate}_{t})(1+\text{Real increase in purchase price}_{t})-1
\end{equation}

\begin{equation}
	\text{Overhead expenses}_{t} = (1 + \text{Inflation rate}_{t})(1+\text{Real increase in overhead expenses}_{t})-1
\end{equation}

\begin{equation}
	\text{Payroll expenses}_{t} = (1 + \text{Inflation rate}_{t})(1+\text{Real increase in payroll expenses}_{t})-1
\end{equation}

\begin{equation}
	\begin{cases}
		\text{Minimum cash required}_{0} = \text{Minimum cash required for initial year} & t=0 \\
		\text{Minimum cash required}_{t} = \% \text{ of sales as cash} \times \text{Total sales}_{t} & t \geq 1
	\end{cases}
\end{equation}

Table 3 Forecasting volume, prices and sales revenues
\begin{equation}
	\text{Increase factor in volume}_{t} = 1 + \text{Increase in sales volume (units)}_{t}, \, t \geq 1
\end{equation}

\begin{equation}
	\begin{cases}
		\text{Sales in units}_{0} = b_{0} \times \text{Price}^{b} & t=0 \\
		\text{Sales in units}_{t} = \text{Sales in units}_{t-1} \times \text{Increase factor in volume}_{t} & t \geq 1
	\end{cases}
\end{equation}

\begin{equation}
	\begin{cases}
		\text{Selling price}_{0} = \text{selling price} & t=0 \\
		\text{Selling price}_{t} = \text{Selling price}_{t-1} \times (1 + \text{selling}_{t}) & t \geq 1
	\end{cases}
\end{equation}

\begin{equation}
	\text{Total sales}_{t} = \text{Selling price}_{t} \times \text{sales in units}_{t}, \, t \geq 1
\end{equation}

Table 4 Forecasting Risk free rate and cost of debt and investment return

\begin{equation}
	\text{Risk free rate}, R_{f,t} = (1 + \text{inflation rate}_{t})(1 + \text{Real interest rate}_{t}) - 1
\end{equation}

\begin{equation}
	\text{Return on ST investment}_{t} = \text{Risk free rate}, R_{f,t} + \text{Risk premium for return on ST investment}_{t}
\end{equation}

\begin{equation}
	\text{Cost of debt}, K_{d,t} = \text{Risk free rate}, R_{f,t} + \text{Risk premium for cost of debt}_{t}
\end{equation}

Table 5 Depreciation schedule and investment in fixed assets
\begin{equation}
	\begin{cases}
		\text{Beginning Net fixed assets}_{0} = 0 & t=0 \\
		\text{Beginning Net fixed assets}_{t} = \text{Beginning Net fixed assets}_{t-1} + \text{New fixed assets}_{t-1} - \text{Annual depreciation}_{t-1} & t\geq1
	\end{cases}
\end{equation}

\begin{equation}
	\begin{cases}
		\text{Annual depreciation for investment in year 0}_{1} = \frac{\text{New fixed assets}_{0}}{\text{Lineal depreciation (4 years)}} & t=1 \\
		\text{Annual depreciation for investment in year 0}_{t} = \text{Annual depreciation for investment in year 0}_{1}&t=2,3,4
	\end{cases}
\end{equation}

\begin{equation}
	\begin{cases}
		\text{Annual depreciation for investment in year 1}_{2} = \frac{\text{New fixed assets}_{1}}{\text{Lineal depreciation (4 years)}} & t=2 \\
		\text{Annual depreciation for investment in year 1}_{t} = \text{Annual depreciation for investment in year 1}_{2}&t=3,4
	\end{cases}
\end{equation}

\begin{equation}
	\begin{cases}
		\text{Annual depreciation for investment in year 2}_{3} = \frac{\text{New fixed assets}_{2}}{\text{Lineal depreciation (4 years)}} & t=3 \\
		\text{Annual depreciation for investment in year 2}_{t} = \text{Annual depreciation for investment in year 2}_{3}&t=4
	\end{cases}
\end{equation}

\begin{equation}
	\text{Annual depreciation for investment in year 3}_{4} = \frac{\text{New fixed assets}_{3}}{\text{Lineal depreciation (4 years)}}
\end{equation}

\begin{equation}
	\begin{cases}
		\text{Annual depreciation}_{0} = 0 & t=0 \\
		\text{Annual depreciation}_{t} = \sum_{k=\max(0,t-4)}^{t-1} \text{Annual depreciation for investment in year } k_{t} & t\geq1
	\end{cases}
\end{equation}

\begin{equation}
	\text{Cumulated depreciation}_{t} = \text{Cumulated depreciation}_{t-1} + \text{Annual depreciation}_{t+1}
\end{equation}
(not available for the last period)

\begin{equation}
	\begin{cases}
		\text{Investment to keep fixed assets constant}_{-1} = \text{Fixed assets}_{0} & t=0 \\
		\text{Investment to keep fixed assets constant}_{t} = \text{Annual depreciation}_{t+1} & t\geq1
	\end{cases}
\end{equation}
(up to second last period)

\begin{equation}
	\text{Investment in fixed assets for growth}_{t} = \text{Net fixed assets}_{t-1} \times \text{Increase in sales volume (units)}_{t+1}
\end{equation}

\begin{equation}
	\text{New fixed assets}_{t} = \text{Investment in fixed assets for growth}_{t} + \text{Investment to keep fixed assets constant}_{t-1}
\end{equation}

\begin{equation}
	\text{Net fixed assets}_{t} = \text{Beginning Net fixed assets}_{t} + \text{New fixed assets}_{t} - \text{Annual depreciation}_{t}
\end{equation}


Table 6a Inventory and purchases in units
\begin{equation}
	\text{Units sold}_{t} = \text{Sales in units}_{t}
\end{equation}

\begin{equation}
	\begin{cases}
		\text{Final inv.}_{0} = \text{Initial investing (units)}_{0} & t=0 \\
		\text{Final inv.}_{t} = \text{Units sold}_{t} \times \text{Inventory as \% of volume in units sold}_{t}&t \geq 1
	\end{cases}
\end{equation}

\begin{equation}
	\text{In. inv.}_{t} = \text{Final inv.}_{t-1}
\end{equation}

\begin{equation}
	\text{Purchases}_{t} = \text{Units sold}_{t} + \text{Final inv.}_{t} - \text{In. inv.}_{t}
\end{equation}

Table 6b Inventory valuation and cost of goods sold in dollars
\begin{equation}
	\begin{cases}
		\text{Unit cost}_{0} = \text{Initial purchase price}_{0} & t=0\\
		\text{Unit cost}_{t} = \text{Unit cost}_{t-1} \times (1+\text{Purchasing}_{t})&t \geq 1
	\end{cases}
\end{equation}

\begin{equation}
	\text{In. inv.}_{t} = \text{Final inv.}_{t-1}
\end{equation}

\begin{equation}
	\text{Purchases}_{t} = \text{Purchases}_{t} \times \text{unit cost}_{t} \quad \text{(purchase on RHS is unit)}
\end{equation}

\begin{equation}
	\text{Final inv.}_{t} = \text{Unit cost}_{t} \times \text{Final inv.}_{t}
\end{equation}

\begin{equation}
	\text{COGS}_{t} = \text{In. inv.}_{t} + \text{Purchases}_{t} - \text{Final inv.}_{t}
\end{equation}

Table 7 Administrative and selling A\&S expenses

\begin{equation}
	\text{Sales commissions}_{t} = \text{Total sales}_{t} \times \text{Selling commissions}_{t-1}
\end{equation}

\begin{equation}
	\begin{cases}
		\text{Overhead expenses}_{0} = \text{Estimated overhead expenses}_{0} & t=0\\
		\text{Overhead expenses}_{t} = \text{Overhead expenses}_{t-1} \times (1+\text{Overhead expenses}_{t}) & t\geq1
	\end{cases}
\end{equation}

\begin{equation}
	\begin{cases}
		\text{Payroll expenses}_{0} = \text{Administrative and sales payroll} & t=0\\
		\text{Payroll expenses}_{t} = \text{Payroll expenses}_{t-1} \times (1+\text{Payroll expenses}_{t}) & t\geq1
	\end{cases}
\end{equation}

\begin{equation}
	\text{Advertising expenses}_{t} = \text{Total sales}_{t} \times \text{Promotion and advertising as a fraction of sales}
\end{equation}

\begin{equation}
	\text{A \& S expenses}_{t} = \text{Sales commissions}_{t} + \text{Overhead expenses}_{t} + \text{Payroll expenses}_{t} + \text{Advertising expenses}_{t}
\end{equation}


Table 8a  Sales and purchases disaggregated according to the timing of flows

\begin{equation}
	\text{Total sales revenues}_{t} = \text{Total sales}_{t}
\end{equation}

\begin{equation}
	\text{Inflow from current year}_{t} = \text{Total sales}_{t} \times \left(1 - \text{Account receivable as \% of sales} - \text{advanced payment from customers as \% of next year sales}\right)
\end{equation}

\begin{equation}
	\text{Credit sales}_{t} = \text{Total sales revenues}_{t} \times \text{Account receivable as \% of sales}
\end{equation}

\begin{equation}
	\text{Payment in advance}_{t} = \text{Total sales revenues}_{t} \times \text{advanced payment from customers as \% of next year sales}
\end{equation}

\begin{equation}
	\text{Total purchases}_{t} = \text{Purchases}_{t}
\end{equation}

\begin{equation}
	\begin{cases}
		\text{Purchases paid the same year}_{0} = \text{Total purchases}_{0} & t=0 \\
		\text{Purchases paid the same year}_{t} = \text{Total purchases} \times \left(1 - \text{Accounts payable as \% of purchases} - \text{Advanced payments from customers as \% of next year sales}\right) & t\geq1
	\end{cases}
\end{equation}

\begin{equation}
	\begin{cases}
		\text{Purchases on credit}_{0} = \text{Purchases}_{0} - \text{Purchases paid the same year}_{0} & t=0 \\
		\text{Purchases on credit}_{t} = \text{Total purchases}_{t} \times \text{Account payable as \% of purchases} & t\geq1
	\end{cases}
\end{equation}

\begin{equation}
	\text{Payment in advance}_{t} = \text{Total purchases}_{t} \times \text{Advance payments to suppliers as a \% of next year purchases}
\end{equation}


Table 8b Flows from sales and purchases
\begin{equation}
	\text{Sales revenues from current year}_{t} = \text{Inflow from current year}_{t}
\end{equation}

\begin{equation}
	\text{Account Receivable}_{t} = \text{Credit sales}_{t-1}
\end{equation}

\begin{equation}
	\text{Advance payments}_{t} = \text{Payment in advance}_{t+1}
\end{equation}

\begin{equation}
	\text{Total inflows}_{t} = \text{Account Receivable}_{t} + \text{Sales revenue from current year}_{t} + \text{Advance payments}_{t}
\end{equation}

\begin{equation}
	\text{Purchases paid the current year}_{t} = \text{Purchases paid the same year}_{t}
\end{equation}

\begin{equation}
	\text{Payment of Accounts Payable}_{t} = \text{Purchases on credit}_{t-1}
\end{equation}


Table 9a Cash budget module 1 operating activities
\begin{equation}
	\text{Inflows from sales}_{t} = \text{Total inflows}_{t}
\end{equation}

\begin{equation}
	\text{Total inflows}_{t} = \text{Inflows from sales}_{t}
\end{equation}

\begin{equation}
	\text{Payments for purchases}_{t} = \text{Total outflows}_{t}
\end{equation}

\begin{equation}
	\text{Administrative and Selling expenses}_{t} = \text{A\&S expenses}_{t}
\end{equation}

\begin{equation}
	\text{Income taxes}_{t} = \text{Income Taxes}_{t}
\end{equation}
(RHS from table 12)

\begin{equation}
	\text{Total cash outflows}_{t} = \text{Payments for purchases}_{t} + \text{Administrative and selling expenses}_{t} + \text{Income Taxes}_{t}
\end{equation}

\begin{equation}
	\text{Operating NCB}_{t} = \text{Total inflows}_{t} - \text{Total cash outflows}_{t}
\end{equation}
net cash balance NCB

Table 9b Cash budget module 2 investing activities
\begin{equation}
	\text{Investment in fixed assets}_{t} = \text{New fixed assets}_{t}
\end{equation}

\begin{equation}
	\text{NCB of investment in assets}_{t} = - \text{Investment in fixed assets}_{t}
\end{equation}

\begin{equation}
	\text{NCB after Capex}_{t} = \text{NCB of investment in assets}_{t} + \text{operating NCB}_{t}
\end{equation}


Table 9c Cash budget module 3 external financing
\begin{equation}
	\text{ST Loan}_{t} = 
	\begin{cases}
		0 & \text{if } \text{cumulated NCB}_{t-1} + \text{Operating NCB}_{t} - \text{Total ST loan payment}_{t} - \text{Minimum cash required}_{t} > 0 \\
		-\left(\text{cumulated NCB}_{t-1} + \text{Operating NCB}_{t} - \text{Total ST loan payment}_{t} - \text{Minimum cash required}_{t}\right) & \text{otherwise}
	\end{cases}
\end{equation}

\begin{equation}
	\text{LT Loan}_{t} = 
	\begin{cases}
		0 & \text{if } \text{cumulated NCB}_{t-1} + \text{NCB after Capex}_{t} + \text{ST Loan}_{t} - \text{Total loan payment}_{t} - \text{Payments to answers}_{t} + \text{Total inflow from ST investment}_{t} - \text{Minimum cash required}_{t} > 0 \\
		-\left(\text{cumulated NCB}_{t-1} + \text{NCB after Capex}_{t} + \text{ST Loan}_{t} - \text{Total loan payment}_{t} - \text{Payments to answers}_{t} + \text{Total inflow from ST investment}_{t} - \text{Minimum cash required}_{t}\right) \times \text{\% of financing with debt, the rest is financed by equity} & \text{otherwise}
	\end{cases}
\end{equation}

\begin{equation}
	\text{Principal ST Loan}_{t} = \text{ST loan}_{t-1}
\end{equation}

\begin{equation}
	\text{Interest ST loan}_{t} = \text{IP}_{t}
\end{equation}
(from Table 11a)

\begin{equation}
	\text{Total ST loan payment}_{t} = \text{Interest ST loan}_{t} + \text{Principal ST loan}_{t}
\end{equation}

\begin{equation}
	\text{Principal LT loan}_{t} = \text{Total PP LT}_{t} \quad \text{(from Table 11b)}
\end{equation}

\begin{equation}
	\text{Interest LT loan}_{t} = \text{Total Interest}_{t} \quad \text{(from Table 11b)}
\end{equation}

\begin{equation}
	\text{Total loan payment}_{t} = \text{Total ST loan payment}_{t} + \text{Principal LT loan}_{t} + \text{Interest LT loan}_{t}
\end{equation}

\begin{equation}
	\text{NCB of financing activities}_{t} = \text{ST loan}_{t} + \text{LT loan}_{t} - \text{Total loan payment}_{t}
\end{equation}


Table 9d Cash budget module 4 transactions with owners
\begin{equation}
	IE_{t} = \frac{LT \, \text{loan}_{t}}{\% \text{ of financing with debt, the rest is financed by equity}} \times \left(1 - \% \text{ of financing with debt, the rest is financed by equity}\right)
\end{equation}

\begin{equation}
	Div_{t} = \text{Next year dividends}_{t-1}
\end{equation}

\begin{equation}
	RS_{t} = \text{Annual depreciation}_{t} \times \text{Stock Repurchases as a \% of depreciation}_{t}
\end{equation}

\begin{equation}
	\text{Payments to owners}_{t} = Div_{t} + RS_{t}
\end{equation}

\begin{equation}
	\text{NCB with owners}_{t} = IE_{t} - \text{Payments to owners}_{t}
\end{equation}

\begin{equation}
	\text{NCB of previous modules}_{t} = \text{NCB with owners}_{t} + \text{NCB of financing activities}_{t} + \text{NCB after Capex}_{t}
\end{equation}

Table 9e Cash budget module 5 discretionary transactions

\begin{equation}
	\text{Redemption of ST investments}_{t} = \text{ST Investments}_{t}
\end{equation}

\begin{equation}
	\text{Return from ST Investments}_{t} = \text{Return on ST Investments}_{t} \times \text{Redemption of ST Investments}_{t}
\end{equation}

\begin{equation}
	\text{Total inflow from ST Investments}_{t} = \text{Return from ST Investments}_{t} + \text{Redemption of ST Investments}_{t}
\end{equation}

\begin{equation}
	\text{ST Investments}_{t} = 
	\begin{cases}
		0 & \text{if } \text{ST loan} + \text{LT loan} + \text{IE} = 0 \\
		\text{Cumulatively NCB}_{t-1} + \text{NCB of previous modules}_{t} + \text{Total inflow from ST Investments}_{t} & \text{otherwise}
	\end{cases}
\end{equation}

\begin{equation}
	\text{NCB of discretionary transactions}_{t} = \text{Total inflow from ST Investments}_{t} - \text{ST Investments}_{t}
\end{equation}

\begin{equation}
	\text{Year NCB}_{t} = \text{NCB at previous modules}_{t} + \text{NCB of discretionary transactions}_{t}
\end{equation}

\begin{equation}
	\text{Cumulated NCB} = \text{Minimum required cash}_{t-1}
\end{equation}



Table 10 Checking the cumulated NCB and the minimum cash target
\begin{equation}
	\begin{cases}
		\text{Calculated Cumulated NCB}_{0} = \text{Year NCB}_{0} &  \\
		\text{Calculated Cumulated NCB}_{t} = \text{Calculated Cumulated NCB}_{t-1} +  \text{Year NCB}_{t} & t \geq 1
	\end{cases}
\end{equation}

\begin{equation}
	\text{Check with MCT} = \text{Calculated Cumulated NCB}_{t} - \text{Minimum cash require}_{t}
\end{equation}


Table 11a Short-term loan schedules
\begin{equation}
	\text{BB}_{t} = \text{EB}_{t-1}
\end{equation}

\begin{equation}
	\text{IP}_{t} = \text{EB}_{t-1} \times \text{Kd}_{t}
\end{equation}

\begin{equation}
	\text{PP}_{t} = \frac{\text{EB}_{t-1}}{\text{Short-term loan 2 (1 year)}}
\end{equation}

\begin{equation}
	\begin{cases}
		\text{EB}_{0} = \text{ST loan}_{0} &  \\
		\text{EB}_{t} = \text{EB}_{t-1} -  \text{PP}_{t} +\text{ST loan}_{t} & t \geq 1
	\end{cases}
\end{equation}

\begin{equation}
	\text{Kd}_{t} = \text{Cost of debt, Kd}_{t}
\end{equation}


Table 11b Long-term loan schedules
\begin{equation}
	\text{BB LT debt}_{t} = \text{EB LT debt}_{t-1}
\end{equation}

\begin{equation}
	\text{LT loan yr 0} = \text{LT loan}_{0}
\end{equation}

\begin{equation}
	\begin{cases}
		\text{PP loan yr 0}_{1} = \frac{\text{LT loan yr 0}}{\text{Long-term (LT) years Loan 3 (M years)}} & t=1 \\
		\text{PP loan yr 0}_{t} = \text{PP loan yr 0}_{1} & t \geq 2
	\end{cases}
\end{equation}

\begin{equation}
	\text{New loan yr 1} = \text{LT loan}_{1}
\end{equation}

\begin{equation}
	\begin{cases}
		\text{PP loan yr 1}_{2} = \frac{\text{New loan yr 1}}{\text{Long-term (LT) years Loan 3 (M years)}} & t = 2 \\
		\text{PP loan yr 1}_{t} = \text{PP loan yr 1}_{2} & t \geq 3
	\end{cases}
\end{equation}

\begin{equation}
	\text{New loan yr 2} = \text{LT loan}_{2}
\end{equation}

\begin{equation}
	\begin{cases}
		\text{PP loan yr 2}_{3} = \frac{\text{New loan yr 2}}{\text{Long-term (LT) years Loan 3 (M years)}} & t = 3 \\
		\text{PP loan yr 2}_{t} = \text{PP loan yr 2}_{3} & t \geq 4
	\end{cases}
\end{equation}

\begin{equation}
	\text{New loan yr 3} = \text{LT loan}_{3}
\end{equation}

\begin{equation}
	\begin{cases}
		\text{PP loan yr 3}_{4} = \frac{\text{New loan yr 3}}{\text{Long-term (LT) years Loan 3 (M years)}} & t = 4 \\
		\text{PP loan yr 3}_{t} = \text{PP loan yr 3}_{4} & t \geq 5
	\end{cases}
\end{equation}

\begin{equation}
	\text{New loan yr 4} = \text{LT loan}_{4}
\end{equation}

\begin{equation}
	\text{Total Interest}_{t} = \text{EB LT debt}_{t-1} \times \text{Kd}_{t} \quad t \geq 1
\end{equation}

\begin{equation}
	\text{New debt LT}_{t} = \text{New loan yr t}
\end{equation}

\begin{equation}
	\text{Total PP LT}_{t} = \sum_{k=0}^{t}\text{PP loan year k}
\end{equation}

\begin{equation}
	\text{EB LT debt}_{t} = \text{BB LT debt}_{t} +  \text{New debt LT}_{t} - \text{Total PP LT}_{t}
\end{equation}


\textbf{Table 12 Income Statement}
\begin{equation}
	\text{Sales revenues}_{t} = \text{Total sales}_{t}
\end{equation}

\begin{equation}
	\text{COGS}_{t} = \text{COGS}_{t}
\end{equation}
RHS from Table 6b

\begin{equation}
	\text{Gross Income}_{t} = \text{Sales revenues}_{t} - \text{COGS}_{t}
\end{equation}

\begin{equation}
	\text{A\&S expenses}_{t} = \text{A\&S expenses}_{t}
\end{equation}
RHS from Table 7

\begin{equation}
	\text{Depreciation}_{t} = \text{Annual depreciation}_{t}
\end{equation}

\begin{equation}
	\text{EBIT}_{t} = \text{Gross Income}_{t} -  \text{A\&S expenses}_{t} - \text{Depreciation}_{t}
\end{equation}

\begin{equation}
	\text{Interest payments}_{t} = \text{Total interest}_{t + \text{IP}_{t}}
\end{equation}

\begin{equation}
	\text{Return from ST investment}_{t} = \text{Return from ST investments}_{t}
\end{equation}
RHS from Table 9e

\begin{equation}
	\text{EBT}_{t} = \text{EBIT}_{t} + \text{Return from ST investment}_{t} - \text{Interest payments}_{t}
\end{equation}

\begin{equation}
	\text{Income Taxes}_{t} = 
	\begin{cases}
		0 & \text{EBT}_{t} \leq 0 \\
		\text{EBT}_{t} \times \text{Corporate tax rate} & \text{otherwise}
	\end{cases}
\end{equation}

\begin{equation}
	 \text{Net Income}_{t} = \text{EBT}_{t} - \text{Income Taxes}_{t}
\end{equation}

\begin{equation}
	 \text{Next year Dividends}_{t} = \text{Net Income}_{t} \times \text{Payout ratio}
\end{equation}

\begin{equation}
	\text{CRE}_{t} = \text{CRE}_{t-1} + \text{Net Income}_{t-1} +  \text{Next year Dividends}_{t-1}
\end{equation}


\textbf{Table 13 The Balance Sheet}
\begin{equation}
	\text{Cash CB}_{t} = \text{Cumulated NCB}_{t}
\end{equation}

\begin{equation}
	\text{AR IT}_{t} = \text{Credit sales}_{t}
\end{equation}

\begin{equation}
	\text{Inventory IT}_{t} = \text{Final inv.}_{t}
\end{equation}

\begin{equation}
	\text{APP}_{t} = \text{Advance payment to suppliers}_{t}
\end{equation}

\begin{equation}
	\text{ST investment CB}_{t} = \text{ST investments.}_{t}
\end{equation}

\begin{equation}
	\text{Current assets}_{t} = \text{Cash CB}_{t} + \text{AR IT}_{t} + \text{Inventory IT}_{t} + \text{APP}_{t} + \text{ST investment CB}_{t}
\end{equation}

\begin{equation}
	\text{Net fixed assets IT}_{t} = \text{Net fixed assets}_{t}
\end{equation}

\begin{equation}
	\text{Total}_{t} = \text{Net fixed assets IT}_{t} + \text{Current assets}_{t}
\end{equation}

\begin{equation}
	\text{AP IT}_{t} = \text{Purchases on credit}_{t}
\end{equation}

\begin{equation}
	\text{APR}_{t} = \text{Advance payments}_{t}
\end{equation}

\begin{equation}
	\text{Short-term debt CB}_{t} = \text{EB}_{t}
\end{equation}

\begin{equation}
	\text{Current liabilities}_{t} = \text{AP IT}_{t} + \text{APR}_{t} + \text{Short-term debt CB}_{t}
\end{equation}

\begin{equation}
	\text{Long-term debt CB}_{t} = \text{EB LT debt}_{t}
\end{equation}

\begin{equation}
	\text{Total Liabilities}_{t} =  \text{Long-term debt CB}_{t} + \text{Current liabilities}_{t}
\end{equation}

\begin{equation}
	\text{Equity investment CB}_{t} = \text{Equity investment CB}_{t-1} + \text{IE}_{t}
\end{equation}

\begin{equation}
	\text{Retained earnings IS}_{t} = \text{CRE}_{t}
\end{equation}

\begin{equation}
	\text{Current year NI}_{t} = \text{Net Income}_{t}
\end{equation}

\begin{equation}
	\text{Repurchase of equity}_{t} = \text{Repurchase of equity}_{t-1} - \text{RS}_{t}
\end{equation}

\begin{equation}
	\text{Liabilities and equity}_{t} = \text{Equity investment CB}_{t} + \text{Retained earnings IS}_{t} + \text{Current year NI}_{t} + \text{Repurchase of equity}_{t} + \text{Total Liabilities}_{t}
\end{equation}

\begin{equation}
	\text{Check}_{t} = \text{Liabilities and equity}_{t} - \text{Total}_{t}
\end{equation}

\end{document}
